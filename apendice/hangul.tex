\section{Anexo A: Alfabeto coreano y su romanización}

Este anexo contiene información adicional sobre el alfabeto coreano y su romanización.

\subsection{Historia del hangul}

El hangul fue creado en 1443 por el rey Sejong el Grande de Corea. El rey Sejong quería crear un sistema de escritura que fuera fácil de aprender y usar para todos los coreanos, independientemente de su nivel de educación o estatus social.

\subsection{Diferentes sistemas de romanización}

Además del sistema McCune-Reischauer, hay otros sistemas de romanización del hangul, como el sistema Yale y el sistema Revised Romanization of Korean (RR).

La siguiente tabla muestra la transliteración de las letras del hangul al sistema McCune-Reischauer:


\begin{table}[t]
	\caption{Romanización McCune-Reischauer}
	\begin{center}
		\begin{tabular}{ | m{2cm} | m{5cm} | m{5cm} | }
			\hline Hangul & McCune-Reischauer \\ \hline
			ㄱ & g \\
			ㄴ & n \\
			ㄷ & d \\
			ㄹ & r \\
			ㅁ & m \\
			ㅂ & b \\
			ㅅ & s \\
			ㅇ & ng \\
			ㅈ & j \\
			ㅊ & ch \\
			ㅋ & k \\
			ㅌ & t \\
			ㅍ & p \\
			ㅎ & h \\
			ㅏ & a \\
			ㅑ & ya \\
			ㅓ & o \\
			ㅕ & yo \\
			ㅗ & o \\
			ㅛ & yo \\
			ㅜ & u \\
			ㅠ & yu \\
			ㅡ & eu \\
			ㅣ & i \\ \hline
		\end{tabular}
	\end{center}
\end{table}



























\subsection{Ejemplos de uso del hangul}

A continuación se muestran algunos ejemplos de cómo se utiliza el hangul en la escritura coreana:

* El nombre del país Corea del Sur se escribe en hangul como 대한민국 (Daehanminguk).
* El nombre de la ciudad de Seúl se escribe en hangul como 서울 (Seoul).
* El nombre de la persona Kim Jong-un se escribe en hangul como 김정은 (Kim Jeong-un).