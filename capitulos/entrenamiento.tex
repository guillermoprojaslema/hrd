\chapter{Programa de Entrenamiento de Hwarangdo\textregistered}

En esta sección, exploraremos el fascinante mundo del Hwarangdo\textregistered, un arte marcial que se caracteriza por su enfoque en las técnicas de combate y la disciplina física y mental. El Hwarangdo\textregistered es conocido por su variedad de movimientos y estilos, y en esta sección, te guiaremos a través de las técnicas clave que componen esta disciplina.

Las técnicas de Hwarangdo\textregistered se dividen en diferentes categorías, cada una con su enfoque particular. A continuación, presentamos una selección de técnicas organizadas de acuerdo a estas categorías para facilitar tu aprendizaje y comprensión:

\begin{enumerate}
	\item Técnicas de golpeo
	\item Técnicas de bloqueo
	\item Técnicas de patadas
	\item Técnicas de agarre y control
	\item Técnicas de defensa personal
\end{enumerate}

A lo largo de esta sección, te proporcionaremos una descripción detallada de cada técnica, así como instrucciones paso a paso sobre cómo realizarlas correctamente. Además, encontrarás consejos y recomendaciones para mejorar tu técnica y tu comprensión de Hwarangdo\textregistered.

Ya seas un principiante que está dando sus primeros pasos en el Hwarangdo\textregistered o un practicante experimentado en busca de refinar sus habilidades, esta sección está diseñada para brindarte información valiosa y recursos que te ayudarán en tu viaje de entrenamiento.

Los practicantes de Hwarangdo\textregistered avanzan a través de niveles de expertise que se distinguen por colores. Cada nivel tiene requisitos específicos que los estudiantes deben cumplir para avanzar al siguiente. A medida que progresas en tu entrenamiento, es fundamental dominar las técnicas correspondientes a tu nivel actual para alcanzar nuevos desafíos y objetivos en tu camino hacia la maestría en Hwarangdo\textregistered.

A continuación, detallaremos los niveles de expertise y las técnicas clave que deben dominar los practicantes para avanzar:


\section{Nivel Blanco}


\subsection[Duración]{Tiempo Estimado}
Tiempo estimado de entrenamiento es de 24 a 36 horas

\subsection{Curso Faja Blanca}
El Curso de faja blanca es el siguiente

\begin{table}[t]
	\caption{Curso Faja Blanca}
	\begin{center}
		\begin{tabular}{ | m{2cm} | m{5cm} | m{5cm} | }
			\hline Detalle & Descripión & Listado\\ \hline
			Soo gi sul dae ryun & Técnicas de puños libres & N°1 al N°5\\
			Kwon bop & Defensa Tradicional & N°1 al N°5\\
			Palkoop sul & Técnicas básicas de codo & N°1 al N°7\\
			Soo Do il bang & Mano de espada/Posición de dragón & N°1 al N°3\\
			Mu Kom sul & Técnica básica espada de madera & N°1 al N° 5\\ \hline
		\end{tabular}
	\end{center}
\end{table}


\subsection{Requerimientos para optar a faja celeste}

\begin{itemize}
	\item Menores de 12 años: 15 técnicas (recreativo - Defensivo)
	\item Mayores de 12 años: Dominar todo el curso de faja blanca
\end{itemize}



Los estudiantes deberán superar un examen para avanzar al siguiente nivel.

\section{Nivel Celeste}

Este color representa el cielo, el empezar a soñar y comenzar la búsqueda para superarse

\subsection[Duración]{Tiempo Estimado}

Tiempo estimado de entrenamiento es de 24 a 36 horas

\subsection{Curso Faja Celeste}
El Curso de faja celeste es el siguiente:

\begin{table}[t]
	\caption{Curso Faja Celeste}
	\begin{center}
		\begin{tabular}{ | m{2cm} | m{5cm} | m{5cm} | }
			\hline Detalle & Descripión & Listado\\ \hline
			Jok sul & Puntapiés básicos & N°8 al N°11\\
			Chayu Matsoki Sul & Técnicas de combate libre & N°1 al N°5\\
			Jok Sul Dae Ryun & Técnicas de puntapiés libres libre & N°1 al N°5\\
			Ho Shin Sul Hwa Rang Do\textregistered & Defensa Personal & N°6 al N°10\\
			Hwarang Kum Sul\textregistered & Técnicas de Espada & Historia\\\hline
		\end{tabular}
	\end{center}
\end{table}

\subsection{Requerimientos para optar a faja Celeste-Amarillo}

\begin{itemize}
	\item Menores de 12 años:
	\item Mayores de 12 años: Dominar el curso de faja Celeste
\end{itemize}

Los estudiantes deberán superar un examen para avanzar al siguiente nivel.

\section{Nivel Celeste-Amarillo}
Este color representa el sol en el cielo, el empezar a comenzar a comprender las técnicas. El amanecer

\subsection[Duración]{Tiempo Estimado}

Tiempo estimado de entrenamiento es de 24 a 36 horas

\subsection{Curso Faja Celeste-Amarillo}
El Curso de faja celeste es el siguiente:

\begin{table}[t]
	\caption{Curso Faja Celeste-Amarillo}
	\begin{center}
		\begin{tabular}{ | m{2cm} | m{5cm} | m{5cm} | }
			\hline Detalle & Descripión & Listado\\ \hline
			Maki sul & Técnicas de Bloqueo Tradicional & N°1 al N°7\\
			Chayu Matsoki Sul & Técnicas de combate libre & N°6 al N°10\\
			Ho Shin Sul Hwa Rang Do\textregistered & Defensa Personal & N°11 al N°15\\
			Chong Du Sul & Técnicas del cammino verdadero & N°1 al N°4\\\hline
			Um Yang Moo Sul & Técnicas militares de endurecimiento & N°1 al N°7\\\hline
		\end{tabular}
	\end{center}
\end{table}

\subsection{Requerimientos para optar a faja Amarillo}

\begin{itemize}
	\item Niños menores de 12 años: 15 técnicas
	\item Mayores de 12 años: Dominio completo del curso faja celeste-amarillo
\end{itemize}

Los estudiantes deberán superar un examen para avanzar al siguiente nivel.

\section{Nivel Amarillo}

Este color representa el sol y la habilidades adquiridas.

\subsection[Duración]{Tiempo Estimado}

Tiempo estimado de entrenamiento es de 24 a 36 horas

\subsection{Requerimientos para optar a faja Amarillo-Verde}

\begin{itemize}
	\item Puñetazos básicos
	\item Bloqueo de ataques simples
	\item Patadas frontales
	\item Postura y equilibrio adecuados
\end{itemize}

Los estudiantes deberán superar un examen para avanzar al siguiente nivel.

\section{Nivel Amarillo-Verde}

Representa el crecimiento de un árbol y a la vez en el alumno sus raíces en el sistema

\subsection[Duración]{Tiempo Estimado}

Tiempo estimado de entrenamiento es de 24 a 36 horas

\subsection{Requerimientos para optar a faja Verde}

\begin{itemize}
	\item Puñetazos básicos
	\item Bloqueo de ataques simples
	\item Patadas frontales
	\item Postura y equilibrio adecuados
\end{itemize}

Los estudiantes deberán superar un examen para avanzar al siguiente nivel.


\section{Nivel Verde}

Representa un árbol alimentándose de agua, nutriendo sus raíces. El alumno se aplica en el conocimiento y desarrollo interno

\subsection[Duración]{Tiempo Estimado}

Tiempo estimado de entrenamiento es de 24 a 36 horas

\subsection{Requerimientos para optar a faja Verde-Azul}

\begin{itemize}
	\item Puñetazos básicos
	\item Bloqueo de ataques simples
	\item Patadas frontales
	\item Postura y equilibrio adecuados
\end{itemize}

Los estudiantes deberán superar un examen para avanzar al siguiente nivel.

\section{Nivel Verde-Azul}

Representa un árbol alimentándose de agua, nutriendo sus raíces. El alumno se aplica en el conocimiento y desarrollo interno

\subsection[Duración]{Tiempo Estimado}

Tiempo estimado de entrenamiento es de 24 a 36 horas

\subsection{Requerimientos para optar a faja Azul}

\begin{itemize}
	\item Puñetazos básicos
	\item Bloqueo de ataques simples
	\item Patadas frontales
	\item Postura y equilibrio adecuados
\end{itemize}

Los estudiantes deberán superar un examen para avanzar al siguiente nivel.

\section{Nivel Azul}

Representa el agua con toda su fuerza. En el alumno aprende a amoldarse a todo tipo de situaciones de conflicto con suavidad o con mucha fuerza

\subsection[Duración]{Tiempo Estimado}

Tiempo estimado de entrenamiento es de 60 horas

\subsection{Requerimientos para optar a faja Roja}

\begin{itemize}
	\item Puñetazos básicos
	\item Bloqueo de ataques simples
	\item Patadas frontales
	\item Postura y equilibrio adecuados
\end{itemize}

Los estudiantes deberán superar un examen para avanzar al siguiente nivel.




¡Comencemos explorando estas emocionantes técnicas de Hwarangdo\textregistered y sumergiéndonos en el mundo de este apasionante arte marcial!