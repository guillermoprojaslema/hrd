\chapter{Vocabulario Coreano-Español}

En este capítulo, exploraremos algunas palabras y frases en coreano junto con sus traducciones al español. A continuación, encontrarás una lista de vocabulario básico.

\section{Saludos}

\begin{itemize}
	\item \textbf{안녕하세요 (Annyeong haseyo)} - Hola.
	\item \textbf{안녕 (Annyeong)} - Hola (informal).
	\item \textbf{안녕히 가세요 (Annyeonghi gaseyo)} - Adiós (cuando alguien se va).
	\item \textbf{안녕히 오세요 (Annyeonghi oseyo)} - Bienvenido (cuando alguien llega).
\end{itemize}

\section{Números}

Existen dos sistemas de numeración en Corea: el sistema nativo y el sistema sino-coreano. El sistema nativo se utiliza en contextos tradicionales y culturales, mientras que el sistema sino-coreano se usa en la mayoría de las situaciones modernas.

\begin{table}[t]
	\caption{Sistema de Numeración}
	\begin{center}
		\begin{tabular}{ | m{2cm} | m{5cm} | m{5cm} | }
			\hline Número & Sino-Coreano & Nativo\\ \hline
			0 & 영, 령 / 공 yeong, ryeong / gong &  \\
			1 & 일 il & 하나 hana \\
			2 & 이 i & 둘 dul \\
			3 & 삼 sam & 셋 set \\
			4 & 사 sa & 넷 net\\
			5 & 오 o & 다섯 daseot\\
			6 & 육 yuk & 여섯 yeoseot\\
			7 & 칠 chil & 일곱 ilgop\\
			8 & 팔 pal & 여덟 yeodeol \\
			9 & 구 gu & 아홉 ahop \\
			10 & 십 sip-il & 열 yeol\\
			11 & 십일 sip-il & 열 yeol-hana \\
			20 & 이십 i-sip & 열 seumul\\
			100 & 백 baek & \\
			1000 & 천 cheon & \\
			10000 & 백 man & \\ \hline
		\end{tabular}
	\end{center}
\end{table}

El uso del sitema Sino-Coreano se utiliza para:

\begin{itemize}
	\item Dar un número de teléfono
	\item Número de habitación
	\item Cálculos
	\item Dinero
	\item Años
	\item Siglos
	\item Número de páginas
\end{itemize}

Mientras que el uso del sistema nativo se usa para

\begin{itemize}
	\item Contar objetos
	\item Contar personas
	\item Decir la edad
	\item Hasta 99
\end{itemize}

\citeA{como-contar-en-coreano}



% Fuente: https://www.monash.edu/arts/languages-literatures-cultures-linguistics/korean-studies-research-hub/research/murksrh-language-lab/topic-1-how-to-know-when-to-use-pure-korean-vs-sino-korean-numbers


\section{Ad-Hoc}

%	경례 %% Saludo
%	검 %% Espada
\begin{itemize}
	\item \textbf{경례 (Hana)} - Saludo.
	\item \textbf{검 (Geom)} - Espada.
	\item \textbf{주먹 (Jumeok)} - Puño.
	\item \textbf{발차기 (Balchagi)} - Patada.
	\item \textbf{바닥 쓸기 (Badak Sseulgi)} - Barrido.
	\item \textbf{주먹질 (Jumeokjil)} - Golpear con el puño.
	\item \textbf{발차기 하다 (Balchagi Hada)} - Hacer una patada.
	\item \textbf{바닥 쓸기 하다 (Badak Sseulgi Hada)} - Realizar un barrido.
	\item \textbf{팔꿈치 (Palkkumchi)} - Codo.
	\item \textbf{무릎 (Mureup)} - Rodilla.
	\item \textbf{머리 (Meori)} - Cabeza.
	\item \textbf{몸 (Mom)} - Cuerpo.
	\item \textbf{방패 (Bangpae)} - Escudo.
	\item \textbf{방어 (Bang-eo)} - Defensa.
	\item \textbf{공격 (Gonggyeok)} - Ataque.
	\item \textbf{전투 (Jeontu)} - Combate.
	\item \textbf{킥 (Kick)} - Patada.
	\item \textbf{킥박스 (Kickbokseu)} - Kickboxing.
	\item \textbf{격투기 (Gyeoktugi)} - Artes marciales.
	\item \textbf{격투수 (Gyeoktu su)} - Luchador.
	\item \textbf{싸움 (Ssaum)} - Pelea.
	\item \textbf{방어 기술 (Bang-eo gisul)} - Técnicas de defensa.
	\item \textbf{공격 기술 (Gonggyeok gisul)} - Técnicas de ataque.
	\item \textbf{도장 (Dojang)} - Lugar de entrenamiento.
	\item \textbf{훈련 (Hunlyeon)} - Entrenamiento.
	\item \textbf{도복 (Dobok)} - Uniforme de artes marciales.
	\item \textbf{격투 무술 (Gyeoktu musul)} - Artes marciales.
	\item \textbf{급 (Geup)} - Grado (como en cinturón).
	\item \textbf{검도 (Geomdo)} - Esgrima.
	\item \textbf{권술 (Gwonsool)} - Defensa personal.

\end{itemize}


\section{Partes del Cuerpo}

\begin{itemize}
	\item \textbf{머리 (Meori)} - Cabeza.
	\item \textbf{목 (Mok)} - Cuello.
	\item \textbf{어깨 (Eokkae)} - Hombro.
	\item \textbf{가슴 (Gaseum)} - Pecho.
	\item \textbf{팔 (Pal)} - Brazo.
	\item \textbf{손 (Son)} - Mano.
	\item \textbf{손목 (Sonmok)} - Muñeca.
	\item \textbf{손가락 (Songarak)} - Dedo.
	\item \textbf{등 (Deung)} - Espalda.
	\item \textbf{허리 (Heori)} - Cintura.
	\item \textbf{배 (Bae)} - Abdomen.
	\item \textbf{다리 (Dari)} - Pierna.
	\item \textbf{무릎 (Mureup)} - Rodilla.
	\item \textbf{발 (Bal)} - Pie.
	\item \textbf{발목 (Balmok)} - Tobillo.
	\item \textbf{발가락 (Balgarak)} - Dedo del pie.
	\item \textbf{머리카락 (Meorikarak)} - Cabello.
	\item \textbf{눈 (Nun)} - Ojo.
	\item \textbf{귀 (Gwi)} - Oído.
	\item \textbf{입 (Ip)} - Boca.
	\item \textbf{이 (I)} - Diente.
	\item \textbf{혀 (Hyeo)} - Lengua.
	\item \textbf{코 (Ko)} - Nariz.
	\item \textbf{입술 (Ipsul)} - Labio.
	\item \textbf{턱 (Teok)} - Barbilla.
\end{itemize}