\chapter{Técnicas del Monje Wong-Gwang Popsa}

El capítulo dedicado a las ``Técnicas del Monje Wong-Gwang Popsa'' representa un componente excepcionalmente valioso en el vasto repertorio de Hwarangdo\textregistered. Este enfoque singular se centra en la capacidad de responder de manera efectiva a un ataque del oponente a través de una combinación de bloqueo, esquiva y desvío de los golpes entrantes. Los practicantes de esta disciplina marcial aprenden a anticipar y contrarrestar las embestidas del oponente con precisión y agilidad.

Una vez que se ha neutralizado el ataque, el defensor responde con una serie de movimientos que pueden incluir proyecciones y retenciones en puntos específicos de articulación o puntos de presión, lo que añade una dimensión única al combate en Hwarangdo\textregistered. La maestría de estas técnicas no solo requiere habilidad física, sino también un profundo conocimiento de la anatomía humana y la capacidad de aprovechar la fuerza y el impulso del oponente en beneficio propio.

Este capítulo no solo imparte habilidades de autodefensa prácticas, sino que también promueve la agudeza mental y la adaptabilidad, ya que los practicantes deben tomar decisiones rápidas en situaciones de combate. La atención al detalle y la concentración son esenciales para el éxito en las Técnicas del Monje Wong-Gwang Popsa, lo que hace que esta área de Hwarangdo\textregistered sea una de las más desafiantes y gratificantes para aquellos que buscan dominar este antiguo arte marcial.

\section{Preámbulo}

\subsection{Técnica 1}

En la Figura \ref{fig:figura1} podemos ver un paisaje antes de la guerra. En la Figura \ref{fig:figura2} se muestra el mismo paisaje después de la guerra. El tratado X se llevó a cabo para resolver el conflicto, como se ilustra en la Figura \ref{fig:figura3}.
\lipsum[1]

%Mostrar una secuencia a 3 fotos por renglón
\begin{figure}[h]
	\centering
	\begin{minipage}{0.3\textwidth}
		\includegraphics[width=\linewidth]{images/Técnicas/puno_cerrado.jpg}
		\caption{Paisaje antes de la guerra}
		\label{fig:figura1}
	\end{minipage}
	\hfill
	\begin{minipage}{0.3\textwidth}
		\includegraphics[width=\linewidth]{images/Técnicas/puno_cerrado.jpg}
		\caption{Paisaje después de la guerra}
		\label{fig:figura2}
	\end{minipage}
	\hfill
	\begin{minipage}{0.3\textwidth}
		\includegraphics[width=\linewidth]{images/Técnicas/puno_cerrado.jpg}
		\caption{Tratado X}
		\label{fig:figura3}
	\end{minipage}
\end{figure}



% Mostrar una secuencia a 2 fotos
%\begin{figure}[h]
%	\centering
%	\begin{minipage}{0.45\textwidth}
%		\includegraphics[width=\linewidth]{images/Técnicas/puno_cerrado.jpg}
%		\caption{Paisaje antes de la guerra}
%		\label{fig:figura1}
%	\end{minipage}
%	\hfill
%	\begin{minipage}{0.45\textwidth}
%		\includegraphics[width=\linewidth]{images/Técnicas/puno_cerrado.jpg}
%		\caption{Paisaje después de la guerra}
%		\label{fig:figura2}
%	\end{minipage}
%	\hfill
%	%	\begin{minipage}{0.3\textwidth}
%	%		\includegraphics[width=\linewidth]{images/Técnicas/puno_cerrado.jpg}
%	%		\caption{Tratado X}
%	%		\label{fig:figura3}
%	%	\end{minipage}
%\end{figure}






